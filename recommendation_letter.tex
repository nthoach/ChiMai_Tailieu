\documentclass[10pt]{article}
\usepackage[utf8]{inputenc}
\usepackage{geometry}
\geometry{margin=0.4in}
\usepackage{amsmath}
\usepackage{graphicx}
\usepackage{color}
\setlength{\parindent}{0pt}
\setlength{\parskip}{0.8em}
\linespread{1.1}

\begin{document}

\textbf{Nguyen The Hoach} \\
Advanced Power and Energy Center, Electrical Engineering Department, Khalifa University \\
Office: 2339, Building 2, SAN Campus \\
Email: nguyen.hoach@ku.ac.ae \\

December 22, 2025

Hiring Committee \\
New York University Abu Dhabi \\
Abu Dhabi, UAE

Dear Hiring Committee,

As a \textbf{Research Scientist} in the Advanced Power and Energy Center at Khalifa University, with research expertise in power electronics, renewable energy systems, and embedded AI for autonomous applications, I am pleased to recommend \textbf{Ms. Safa Mohammed Sali} for the Research Assistant position at New York University Abu Dhabi. I have known \textbf{Ms. Safa} since she joined Khalifa University for her internship in June 2024, where I served as her co-advisor and mentor on research projects in power electronics and embedded systems, alongside Dr. Ameena Saad Al-Sumaiti.

\textbf{Ms. Safa} has demonstrated exceptional research capabilities, consistently delivering high-quality work on time in both theoretical analysis and laboratory prototyping. Her contributions to our collaborative projects have shown her proficiency in advanced engineering tools and methodologies.

\textbf{Ms. Safa} holds B.Tech. and M.Tech. degrees in Electronics \& Communication Engineering and Embedded Systems, respectively, from APJ Abdul Kalam Technological University. She has achieved an IELTS score of 7.0 and GRE scores of 151 in Quantitative Reasoning, 137 in Verbal Reasoning, and 3.0 in Writing, underscoring her solid foundation in science and technology. Her academic English skills are evident in our co-authored papers and her independent publications.

In our collaboration on power electronics, \textbf{Ms. Safa} designed and simulated high-gain voltage-multiplier coupled quadratic boost converters for small-scale PV integration, performing CCM analysis, gain derivation, component sizing, and closed-loop PI voltage control in MATLAB/Simulink. This work aligns with my research in advanced power conversion topologies and demonstrates her ability to handle complex engineering challenges.

Beyond power electronics, \textbf{Ms. Safa's} work in FPGA-based embedded AI complements my interests in hardware acceleration for real-time systems. She has independently developed projects on Xilinx platforms, including real-time object detection on Kria KV260 using Vitis AI, where she converted and quantized models (YOLOv3/SSD to INT8), integrated runtimes with video I/O, and benchmarked throughput and latency. She also developed compressed-domain tracking pipelines using H.264 motion vectors for autonomous systems, showcasing skills in PS-PL integration, AXI protocols, and resource optimization.

\textbf{Ms. Safa} has prepared well for a research-oriented career by publishing several papers, including submissions to IEEE Transactions on Intelligent Transportation Systems, IEEE Access, and arXiv preprints on FPGA-based CNNs, transformers, and AI scheduling. Her work includes reviews on hardware acceleration in MRIs and autonomous vehicles, reflecting her ability to conduct rigorous, publishable research.

In my professional capacity, I have supervised numerous interns and graduate students, publishing extensively in IEEE journals on power electronics and embedded systems. \textbf{Ms. Safa's} performance stands out, and I am confident she will excel in a research assistant role, contributing meaningfully to NYU Abu Dhabi's initiatives in engineering and technology.

I wholeheartedly recommend \textbf{Ms. Safa Mohammed Sali} for this position. Her technical skills, research experience, and dedication make her an ideal candidate.

Yours sincerely,

\textbf{Dr. Nguyen The Hoach} \\
Research Scientist \\
Advanced Power and Energy Center \\
Khalifa University

\end{document}